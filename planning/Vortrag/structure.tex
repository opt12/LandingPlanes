%----------------------------------------------------------------------------------------
%	REQUIRED PACKAGES
%----------------------------------------------------------------------------------------

\usepackage{graphicx} % Required for including images
\graphicspath{{Figures/}} % Set the default folder for images

\usepackage[square,
authoryear,
]{natbib}

\usepackage{relsize, etoolbox, lmodern}% http://ctan.org/pkg/{relsize,etoolbox}

\AtBeginEnvironment{quote}{\smaller\fontfamily{lmss}\selectfont}% Step font down one size relative to current font.
% see http://tex.stackexchange.com/questions/25249/how-do-i-use-a-particular-font-for-a-small-section-of-text-in-my-document for fontfamilies

\usepackage{chngcntr}

% to have a separation in the text with extra space between paragraphs and no indendation following
\newcommand*{\skippingparagraph}{\par\vspace{1.0\baselineskip}\noindent}

\usepackage[absolute,overlay]{textpos}

\setbeamercolor{framesource}{fg=gray}
\setbeamerfont{framesource}{size=\tiny}

\newcommand{\source}[1]{\begin{textblock*}{6cm}(6.7cm,8.6cm)
		\begin{beamercolorbox}[ht=0.5cm,right]{framesource}
			\usebeamerfont{framesource}\usebeamercolor[fg]{framesource} Source: {#1}
		\end{beamercolorbox}
\end{textblock*}}


%Framenumber in Header
%http://tex.stackexchange.com/a/145899/123262

\makeatletter
\setbeamertemplate{frametitle}{%
	\nointerlineskip%
	\vskip-\beamer@headheight%
	\vbox to \beamer@headheight{%
		\vfil
		\leftskip=-\beamer@leftmargin%
		\advance\leftskip by0.3cm%
		\rightskip=-\beamer@rightmargin%
		\advance\rightskip by0.3cm plus1fil%
		{\usebeamercolor[fg]{frametitle}
			\usebeamerfont{frametitle}\insertframetitle\par}% added number
		{\usebeamercolor[fg]{framesubtitle}
			\usebeamerfont{framesubtitle}\insertframesubtitle\hfill{\tiny\insertframenumber/\inserttotalframenumber}\par}%
		\vbox{}%
		\vskip-1em%
		\vfil
	}%
}
\makeatother
%----------------------------------------------------------------------------------------
%	HYPERLINKS
%---------------------------------------------------------------------------------------

\hypersetup{
	%draft, % Uncomment to remove all links (useful for printing in black and white)
	colorlinks=true, breaklinks=true, %bookmarks=bookmarksnumbered,
	urlcolor=brown, linkcolor=blue, citecolor=green, % Link colors
	pdftitle={}, % PDF title
	pdfauthor={\textcopyright}, % PDF Author
	pdfsubject={}, % PDF Subject
	pdfkeywords={}, % PDF Keywords
	pdfcreator={pdfLaTeX}, % PDF Creator
	pdfproducer={LaTeX with hyperref and ClassicThesis} % PDF producer
}


%----------------------------------------------------------------------------------------
%	CODESNIPPETS
%---------------------------------------------------------------------------------------

%%%%%%%%%%%%%%%%%%%%%%%%%%%%%%%%%%%%%%%%%
% Code Snippet
% LaTeX Template
% Version 1.0 (14/2/13)
%
% This template has been downloaded from:
% http://www.LaTeXTemplates.com
%
% Original author:
% Velimir Gayevskiy (vel@latextemplates.com)
%
% License:
% CC BY-NC-SA 3.0 (http://creativecommons.org/licenses/by-nc-sa/3.0/)
%
%%%%%%%%%%%%%%%%%%%%%%%%%%%%%%%%%%%%%%%%%
\usepackage{lmodern}
\usepackage{textcomp}
\usepackage{wasysym}

\usepackage{listings} % Required for inserting code snippets
\usepackage{pgf, pgffor}
\usepackage{lstlinebgrd} % see http://www.ctan.org/pkg/lstaddons

\definecolor{DarkGreen}{rgb}{0.0,0.4,0.0} % Comment color
\definecolor{highlight}{RGB}{255,251,204} % Code highlight color

\usepackage{savesym}	%see http://www.tex.ac.uk/FAQ-alreadydef.html why we need this stunt here
\savesymbol{Square}
\usepackage{bbding}
\restoresymbol{BBDing}{Square}

\usepackage{tikz} % https://tex.stackexchange.com/a/13095

%%% listingsbackground from http://tex.stackexchange.com/a/85628/123262
\makeatletter
%%%%%%%%%%%%%%%%%%%%%%%%%%%%%%%%%%%%%%%%%%%%%%%%%%%%%%%%%%%%%%%%%%%%%%%%%%%%%%
%
% \btIfInRange{number}{range list}{TRUE}{FALSE}
%
% Test in int number <number> is element of a (comma separated) list of ranges
% (such as: {1,3-5,7,10-12,14}) and processes <TRUE> or <FALSE> respectively

\newcount\bt@rangea
\newcount\bt@rangeb

\newcommand\btIfInRange[2]{%
	\global\let\bt@inrange\@secondoftwo%
	\edef\bt@rangelist{#2}%
	\foreach \range in \bt@rangelist {%
		\afterassignment\bt@getrangeb%
		\bt@rangea=0\range\relax%
		\pgfmathtruncatemacro\result{ ( #1 >= \bt@rangea) && (#1 <= \bt@rangeb) }%
		\ifnum\result=1\relax%
		\breakforeach%
		\global\let\bt@inrange\@firstoftwo%
		\fi%
	}%
	\bt@inrange%
}
\newcommand\bt@getrangeb{%
	\@ifnextchar\relax%
	{\bt@rangeb=\bt@rangea}%
	{\@getrangeb}%
}
\def\@getrangeb-#1\relax{%
	\ifx\relax#1\relax%
	\bt@rangeb=100000%   \maxdimen is too large for pgfmath
	\else%
	\bt@rangeb=#1\relax%
	\fi%
}

%%%%%%%%%%%%%%%%%%%%%%%%%%%%%%%%%%%%%%%%%%%%%%%%%%%%%%%%%%%%%%%%%%%%%%%%%%%%%%
%
% \btLstHL<overlay spec>{range list}
%
% TODO BUG: \btLstHL commands can not yet be accumulated if more than one overlay spec match.
% 
\newcommand<>{\btLstHL}[1]{%
	\only#2{\btIfInRange{\value{lstnumber}}{#1}{\color{orange!30}\def\lst@linebgrdcmd{\color@block}}{\def\lst@linebgrdcmd####1####2####3{}}}%
}%
\makeatother

\lstdefinestyle{JSStyle}{ % Define a style for your code snippet, multiple definitions can be made if, for example, you wish to insert multiple code snippets using different programming languages into one document
	language=JavaScript, % Detects keywords, comments, strings, functions, etc for the language specified
%	backgroundcolor=\color{highlight}, % Set the background color for the snippet - useful for highlighting
	basicstyle=\smaller\smaller\ttfamily, % The default font size and style of the code
	breakatwhitespace=true, % If true, only allows line breaks at white space
%	breaklines=true, % Automatic line breaking (prevents code from protruding outside the box)
	captionpos=b, % Sets the caption position: b for bottom; t for top
	commentstyle=\usefont{T1}{pcr}{m}{sl}\color{DarkGreen}, % Style of comments within the code - dark green courier font
	deletekeywords={}, % If you want to delete any keywords from the current language separate them by commas
	%escapeinside={\%}, % This allows you to escape to LaTeX using the character in the bracket
	firstnumber=1, % Line numbers begin at line 1
%	frame=single, % Frame around the code box, value can be: none, leftline, topline, bottomline, lines, single, shadowbox
%	frameround=tttt, % Rounds the corners of the frame for the top left, top right, bottom left and bottom right positions
	keywordstyle=\color{Blue}\textbf, % Functions are bold and blue
	morekeywords={}, % Add any functions no included by default here separated by commas
	numbers=left, % Location of line numbers, can take the values of: none, left, right
	numbersep=5pt, % Distance of line numbers from the code box
	numberstyle=\tiny\color{Gray}, % Style used for line numbers
	rulecolor=\color{black}, % Frame border color
	showstringspaces=false, % Don't put marks in string spaces
	showtabs=false, % Display tabs in the code as lines
%	stepnumber=5, % The step distance between line numbers, i.e. how often will lines be numbered
	stringstyle=\color{Purple}, % Strings are purple
	tabsize=2, % Number of spaces per tab in the code
}

\lstdefinestyle{myCStyle}{ % Define a style for your code snippet, multiple definitions can be made if, for example, you wish to insert multiple code snippets using different programming languages into one document
	language=myC, % Detects keywords, comments, strings, functions, etc for the language specified
%	backgroundcolor=\color{highlight}, % Set the background color for the snippet - useful for highlighting
	basicstyle=\smaller\smaller\ttfamily, % The default font size and style of the code
	breakatwhitespace=true, % If true, only allows line breaks at white space
%	breaklines=true, % Automatic line breaking (prevents code from protruding outside the box)
	captionpos=b, % Sets the caption position: b for bottom; t for top
	commentstyle=\usefont{T1}{pcr}{m}{sl}\color{DarkGreen}, % Style of comments within the code - dark green courier font
	deletekeywords={}, % If you want to delete any keywords from the current language separate them by commas
	%escapeinside={\%}, % This allows you to escape to LaTeX using the character in the bracket
	firstnumber=1, % Line numbers begin at line 1
%	frame=single, % Frame around the code box, value can be: none, leftline, topline, bottomline, lines, single, shadowbox
%	frameround=tttt, % Rounds the corners of the frame for the top left, top right, bottom left and bottom right positions
	keywordstyle=\color{Blue}\textbf, % Functions are bold and blue
	morekeywords={}, % Add any functions no included by default here separated by commas
	numbers=left, % Location of line numbers, can take the values of: none, left, right
	numbersep=5pt, % Distance of line numbers from the code box
	numberstyle=\tiny\color{Gray}, % Style used for line numbers
	rulecolor=\color{black}, % Frame border color
	showstringspaces=false, % Don't put marks in string spaces
	showtabs=false, % Display tabs in the code as lines
%	stepnumber=5, % The step distance between line numbers, i.e. how often will lines be numbered
	stringstyle=\color{Purple}, % Strings are purple
	tabsize=2, % Number of spaces per tab in the code
}

\definecolor{darkgray}{rgb}{.4,.4,.4}
\definecolor{purple}{rgb}{0.65, 0.12, 0.82}


%define Javascript language
\lstdefinelanguage{JavaScript}{
	keywords={typeof, new, true, false, catch, try, finally, function, return, null, then, catch, switch, var, if, in, while, do, else, case, break, yield, async, await, let, const},
	keywordstyle=\color{blue}\bfseries,
	ndkeywords={class, export, boolean, throw, implements, import, this},
	ndkeywordstyle=\color{darkgray}\bfseries,
	identifierstyle=\color{black},
	sensitive=false,
	comment=[l]{//},
	morecomment=[s]{/*}{*/},
	commentstyle=\color{purple}\ttfamily,
	stringstyle=\color{red}\ttfamily,
	morestring=[b]',
	morestring=[b]"
}

\lstdefinelanguage{myC}{
	basicstyle=\footnotesize\sffamily\color{black},
	keywords={struct, int, float, size\_t, const, if, else , return},
	identifierstyle=\color{black},
	sensitive=false,
	comment=[l]{//},
	morecomment=[s]{/*}{*/},
	keywordstyle=\color{blue}\bfseries,
	showspaces=false,
	showstringspaces=false,
	commentstyle=\color{purple}\ttfamily,
	stringstyle=\color{red}\ttfamily,
	morestring=[b]",
	breaklines=true,
	postbreak=\mbox{\textcolor{red}{$\hookrightarrow$}\space}
}


\lstset{
	language=JavaScript,
	extendedchars=true,
	basicstyle=\footnotesize\ttfamily,
	showstringspaces=false,
	showspaces=false,
	numbers=left,
	numberstyle=\footnotesize,
	numbersep=9pt,
	tabsize=2,
	breaklines=true,
	showtabs=false,
	captionpos=b,
	escapeinside={(*@}{@*)},	%see http://tex.stackexchange.com/a/8860/123262
}

% Create a command to cleanly insert a snippet with the style above anywhere in the document
\newcommand{\insertcode}[1]{\lstinputlisting[label=#1,style=JSStyle]{#1}} % The first argument is the script location/filename

