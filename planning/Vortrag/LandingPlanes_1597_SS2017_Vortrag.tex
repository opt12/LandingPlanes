% $Header$

\documentclass[
	xcolor=dvipsnames,
	hideallsubsections
]{beamer} %xcolor=dvipsnames
%\documentclass[xcolor=dvipsnames, handout]{beamer} %xcolor=dvipsnames

\usepackage{pgfpages}

\mode<presentation>
{
	\usetheme{PaloAlto}
	\usecolortheme[named=TealBlue]{structure}
%	\usecolortheme[named=Dandelion]{structure}
	
	\setbeamercovered{transparent}
	% or whatever (possibly just delete it)
%	\setbeameroption{show notes} % un-comment to see the notes
%	\setbeameroption{show notes on second screen=right}
}


%%%%%%%%%%%%%%%%%%%%%%%%%%%%%%%%%%%%%%%%%%%%%%%%%%%%%%%%%%%%%%%%%%%%%%%%%%%%%%%%%%%%%%%%%
% This LaTeX Beamer file is based on the solution template for:

% - Giving a talk on some subject.
% - The talk is between 15min and 45min long.
% - Style is ornate.

% Copyright 2004 by Till Tantau <tantau@users.sourceforge.net>.
%
% In principle, this file can be redistributed and/or modified under
% the terms of the GNU Public License, version 2.
%
% However, this file is supposed to be a template to be modified
% for your own needs. For this reason, if you use this file as a
% template and not specifically distribute it as part of a another
% package/program, I grant the extra permission to freely copy and
% modify this file as you see fit and even to delete this copyright
% notice. 
%%%%%%%%%%%%%%%%%%%%%%%%%%%%%%%%%%%%%%%%%%%%%%%%%%%%%%%%%%%%%%%%%%%%%%%%%%%%%%%%%%%%%%%%%

% Thank you so much Till :-)


\usepackage[ngerman]{babel}
% or whatever

\usepackage[latin1]{inputenc}
% or whatever

\usepackage{times}
\usepackage[T1]{fontenc}
 
\input{structure.tex} % Include the structure.tex file which specified the document structure and layout


% Or whatever. Note that the encoding and the font should match. If T1
% does not look nice, try deleting the line with the fontenc.

\title[Notlandefelder aus H�hendaten] % (optional, use only with long paper titles)
{Erkennung von Notlandefeldern aus H�hendaten: \\
	Parallele Implementierung der Durchmusterung mit \texttt{pThreads}}

\author[] % (optional, use only with lots of authors)
{Felix~Eckstein, Dr. Bj�rn Wittich}
% - Use the \inst{?} command only if the authors have different
%   affiliation.

%\institute[FernUni Hagen] % (optional, but mostly needed)
%{
%}
  
% - Use the \inst command only if there are several affiliations.
% - Keep it simple, no one is interested in your street address.

\date[] % (optional)
{``1597 -- Fachpraktikum Parallele Programmierung''\\
	im Sommersemester 2017 an der FernUniversit�t Hagen\\
23. September 2017}

\subject{Talks}
% This is only inserted into the PDF information catalog. Can be left
% out. 



% If you have a file called "university-logo-filename.xxx", where xxx
% is a graphic format that can be processed by latex or pdflatex,
% resp., then you can add a logo as follows:

% \pgfdeclareimage[height=1cm]{university-logo}{images/HannoverJs.png}
  \pgfdeclareimage[height=1cm]{university-logo}{images/UniHagen.png}
 \logo{\pgfuseimage{university-logo}}



% Delete this, if you do not want the table of contents to pop up at
% the beginning of each subsection:
\AtBeginSection[]
{
  \begin{frame}<beamer>{Outline}
%    \tableofcontents[currentsection]%,currentsubsection]
    \tableofcontents[sections=\thesection]
  \end{frame}
}

\begin{document}

\begin{frame}
  \titlepage
	\note{
		\begin{itemize}
			\item In den \textbackslash note Abschnitt kommen Vortragsnotizen
			\item \ldots
		\end{itemize}
			
	Kein Vortrag ohne Katzenfotos!
	}
\end{frame}



\begin{frame}{Outline}
  	\tableofcontents[hideallsubsections]
	\note{
		\begin{itemize}
			\item In den \textbackslash note Abschnitt kommen Vortragsnotizen
			\item \ldots
		\end{itemize}
			
	Kein Vortrag ohne Katzenfotos!
	}
\end{frame}


% Since this a solution template for a generic talk, very little can
% be said about how it should be structured. However, the talk length
% of between 15min and 45min and the theme suggest that you stick to
% the following rules:  

% - Exactly two or three sections (other than the summary).
% - At *most* three subsections per section.
% - Talk about 30s to 2min per frame. So there should be between about
%   15 and 30 frames, all told.

\setcounter{framenumber}{\value{framenumber}-1}	%no idea why it otherwise skips one page
\section{Aufgabenstellung}

\begin{frame}{Funktionsaufrufe in Single-Threaded-Umgebungen}{``Synchrone Programmierung''}
	
	\begin{columns}
		\column{0.5\textwidth}
		\begin{itemize}[<+->]
			\item Synchrone Funktionen
			\begin{itemize}[<.->]
				\item werden aufgerufen
				\item berechnen ihr Ergebnis
				\item geben das Resultat zur�ck
			\end{itemize}
			\item Funktionsaufrufe k�nnen die Ausf�hrung blockieren
			\item Das Warten auf externe Ereignisse f�hrt zu Ineffizienz
			\item Es ist keine Neben- l�ufigkeit m�glich
		\end{itemize}
		\column{0.5\textwidth}
			\begin{figure}
				\includegraphics[height=.8\textheight]{images/syncFlow.png}
			\end{figure}
	\end{columns}

	\note{
		\begin{itemize}
			\item Das Hauptprogramm kann nicht weiterarbeiten
			\item Nebenl�ufigkeit ist nicht m�glich
			
			\skippingparagraph
			
			\skippingparagraph
			
			\item Was l�sst sich daran besser machen?
		\end{itemize}
	}

\end{frame}

	\begin{frame}{Asynchrone Programmierung}
		
	\begin{itemize}[<+->]
		\item<1-> Asynchrone Funktionen verhalten sich anders:	
		\begin{itemize}[<+->]
			\item<+-> Async Funktionen werden \alert{jetzt} aufgerufen, \ldots
			\item<+-> \ldots kehren direkt zur�ck\ldots
			\item<+-> \ldots und berechnen ihr Ergebnis
			\item<+-> Diese Ergebnis wird \alert{sp�ter} �bergeben
		\end{itemize}
		\item<+-> Der aufrufende Code erh�lt die Kontrolle direkt zur�ck
		\item<+-> Das Hauptprogramm kann sinnvolle Dinge tun, w�hrend die async Funktion noch ihr Ergebnis ermittelt
		\item<+-> Dadurch gibt es \alert{Nebenl�ufigkeit} auch in single-threaded Umgebungen
	\end{itemize}

	\note{
		Erkl�re hier, dass die asynchronen Funktionen eben \alert{doch in einem anderen Thread} ablaufen k�nnen.

		Eventuell sogar noch die Grafik Zeigen: 
		
		\begin{figure}
			\includegraphics[height=.5\textheight]{images/syncFlow.png}
			\caption{Asnychronous Functions are executed outside the JavaScript-Single-Thread}
		\end{figure}
		
	}
	\end{frame}
	
\section{Architektur}
\subsection{�berblick}
\subsection{Grobk�rnige Parallelit�t}
\subsection{Feingranulare Parallelit�t}

\section{Komponente "`searchEngine"'}

\subsection{Datenformate und Verarbeitung}
\subsubsection{GeoTIFF}
\subsubsection{Speicherformat: GeoJSON}
\subsection{GeoTIFFHandler}
\subsubsection{Datenextraktion}
\subsubsection{Koordinatensysteme}
\subsubsection{Das \texttt{GeoTIFFHandler}-Objekt}


\section{Komponente "'Durchmusterung"'}
\subsection{Durchmusterung der Geotiff Daten und Auffinden der Landebahnen}

\subsection{Interface der Library}

\subsection{Logischer Ablauf innerhalb der Landebahn-Erkennung}

\subsection{Initialisierung}

\subsection{Parallelverarbeitung mittels p\_threads zum Finden der Landebahnen}

\subsection{Detaillierte Beschreibung des Suchalgorithmus}

\subsection{Klassen und Objekte}

\subsection{Verwendete Datenstrukturen}

\subsubsection{Einfluss der Datenstrukturen auf die Performanz}

\subsection{Bestimmung des Speedups in Abh�ngigkeit des Parallelisierungsgrades}

\subsubsection{Erl�uterung des Messverfahren}

\subsubsection{Messergebnisse}

\subsubsection{Diskussion der Messergebnisse}

\subsection{Analyse mittels Profiler}

\subsection{Einfluss von Compiler Direktiven (speziell Optimizer)}

\section{Komponente "`databaseManager"'}

\subsection{Datenbankauswahl}

	\begin{frame}{Ein GeoJSON Objekt zur Ablage in die Datenbank}{}
			
		\smaller\lstinputlisting[style=JSStyle, linebackgroundcolor={%
			\btLstHL<2|handout:0>{18}%
			\btLstHL<3|handout:0>{19,20}% 
		}]{snippets/databaseObject.js}
		
	\end{frame}
	


	\subsection{Express-Server App}
	
	\subsection{Postprocessing - Merge}
	
	\section{Komponente "'landingClient"'}
	
	\begin{frame}{Benutzeroberfl�che}
		\begin{figure}[ht]
			\includegraphics[width=\textwidth]{../Bericht/drawings/LandingClient_Screen1.png}
			\caption{Beispiel Screenshot aus der Weboberfl�che zur Bedienung des Systems}
		\end{figure}
	\end{frame}
	
	\subsection{�berblick}
	
	\subsection{Features und Bedienung}
	
	\section{Ausblick und weitere Ideen}
	
	\subsection{Parallele Bearbeitung mehrerer Kacheln per MPI}
	
	\subsection{Richtungskorrektur des GeoTIFF}
	
	\subsection{Push f�r Webclient}
	
	\subsection{Merge als nachgeschalteter Prozess aus dem Speicher}
	
	\subsection{Kollisionsabfrage mit Objekten aus OSM}
	
\end{document}


