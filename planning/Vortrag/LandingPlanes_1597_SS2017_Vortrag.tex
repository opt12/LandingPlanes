 % $Header$

\documentclass[
	xcolor=dvipsnames,
%	handout,
	hideallsubsections
]{beamer} %xcolor=dvipsnames
%\documentclass[xcolor=dvipsnames, handout]{beamer} %xcolor=dvipsnames

\usepackage{pgfpages}

\mode<presentation>
{
	\usetheme{Rochester}
	\usecolortheme{wolverine}
%	\usecolortheme[named=TealBlue]{structure}
%	\usecolortheme[named=Dandelion]{structure}
	
	\setbeamercovered{transparent}
	% or whatever (possibly just delete it)
%	\setbeameroption{show notes} % un-comment to see the notes
%	\setbeameroption{show notes on second screen=right}
}


%%%%%%%%%%%%%%%%%%%%%%%%%%%%%%%%%%%%%%%%%%%%%%%%%%%%%%%%%%%%%%%%%%%%%%%%%%%%%%%%%%%%%%%%%
% This LaTeX Beamer file is based on the solution template for:

% - Giving a talk on some subject.
% - The talk is between 15min and 45min long.
% - Style is ornate.

% Copyright 2004 by Till Tantau <tantau@users.sourceforge.net>.
%
% In principle, this file can be redistributed and/or modified under
% the terms of the GNU Public License, version 2.
%
% However, this file is supposed to be a template to be modified
% for your own needs. For this reason, if you use this file as a
% template and not specifically distribute it as part of a another
% package/program, I grant the extra permission to freely copy and
% modify this file as you see fit and even to delete this copyright
% notice. 
%%%%%%%%%%%%%%%%%%%%%%%%%%%%%%%%%%%%%%%%%%%%%%%%%%%%%%%%%%%%%%%%%%%%%%%%%%%%%%%%%%%%%%%%%

% Thank you so much Till :-)


\usepackage[ngerman]{babel}
% or whatever

\usepackage[latin1]{inputenc}
% or whatever

\usepackage{times}
\usepackage[T1]{fontenc}
 
\input{structure.tex} % Include the structure.tex file which specified the document structure and layout


% Or whatever. Note that the encoding and the font should match. If T1
% does not look nice, try deleting the line with the fontenc.

\title[Notlandefelder aus H�hendaten] % (optional, use only with long paper titles)
{Erkennung von Notlandefeldern aus H�hendaten: \\
	Parallele Implementierung der Durchmusterung mit \texttt{pThreads}}

\author[] % (optional, use only with lots of authors)
{Felix~Eckstein, Dr. Bj�rn Wittich}
% - Use the \inst{?} command only if the authors have different
%   affiliation.

%\institute[FernUni Hagen] % (optional, but mostly needed)
%{
%}
  
% - Use the \inst command only if there are several affiliations.
% - Keep it simple, no one is interested in your street address.

\date[] % (optional)
{``1597 -- Fachpraktikum Parallele Programmierung''\\
	im Sommersemester 2017 an der FernUniversit�t Hagen\\
23. September 2017}

\subject{Talks}
% This is only inserted into the PDF information catalog. Can be left
% out. 



% If you have a file called "university-logo-filename.xxx", where xxx
% is a graphic format that can be processed by latex or pdflatex,
% resp., then you can add a logo as follows:

% \pgfdeclareimage[height=1cm]{university-logo}{images/HannoverJs.png}
  \pgfdeclareimage[height=1cm]{university-logo}{images/UniHagen.png}
 \logo{\pgfuseimage{university-logo}}



% Delete this, if you do not want the table of contents to pop up at
% the beginning of each subsection:
\AtBeginSection[]
{
  \begin{frame}<beamer> {Outline}
%    \tableofcontents[currentsection]%,currentsubsection]
    \tableofcontents[sections=\thesection]
  \end{frame}
}

\begin{document}

\begin{frame}
  \titlepage
	\note{
		\begin{itemize}
			\item In den \textbackslash note Abschnitt kommen Vortragsnotizen
			\item \ldots
		\end{itemize}
			
	Kein Vortrag ohne Katzenfotos!
	}
\end{frame}



\begin{frame} {Outline}
  	\tableofcontents[hideallsubsections]
	\note{
		\begin{itemize}
			\item In den \textbackslash note Abschnitt kommen Vortragsnotizen
			\item \ldots
		\end{itemize}
			
	Kein Vortrag ohne Katzenfotos!
	}
\end{frame}


% Since this a solution template for a generic talk, very little can
% be said about how it should be structured. However, the talk length
% of between 15min and 45min and the theme suggest that you stick to
% the following rules:  

% - Exactly two or three sections (other than the summary).
% - At *most* three subsections per section.
% - Talk about 30s to 2min per frame. So there should be between about
%   15 and 30 frames, all told.

\setcounter{framenumber}{\value{framenumber}-1}	%no idea why it otherwise skips one page
\section{Aufgabenstellung}

\begin{frame} {Title}{Subtitle}
	


	\note{
		\begin{itemize}
			\item 
		\end{itemize}
	}

\end{frame}

\begin{frame} {titel}
	
\begin{itemize}[<+->]
	\item<1-> 
	\begin{itemize}[<+->]
		\item<+-> 
		\item<+-> 
		\item<+-> 
		\item<+-> 
	\end{itemize}
	\item<+-> 
\end{itemize}

\note{
	
}
\end{frame}
	
\section{Architektur}

\subsection{Architekturziele}

\begin{frame} {Systemarchitektur}{Anforderungen}
	
	Anforderungen an die Architektur:
	
	\begin{itemize}
	\item inh�rent parallel
	\item lose gekoppelte Komponenten
	\item einfache Einbindung externer Bibliotheken
	\item einfache Erweiterbarkeit
	\item Kommunikation �ber Standardprotokolle
	\end{itemize}
	
	\note{
		\begin{itemize}
			\item 
		\end{itemize}
	}

\end{frame}

\subsection{�berblick}

\begin{frame} {Systemarchitektur}{�berblick}
	
	Die Architektur implementiert ein MVW (Model View Whatever) Modell:
	
	\begin{figure}
		\includegraphics[height=.8\textheight]{images/Architektur.png}
	\end{figure}

	\note{
		\begin{itemize}
			\item Model: Datenbank
			\item View: userInterface
			\item Whatever hier:
			\begin{itemize}
			\item Controller: DataBaseManager
			\item Businnes Logic: searchEngine
			\end{itemize}
		\end{itemize}
	}

\end{frame}

\subsection{Komponenten}

\begin{frame} {Systemarchitektur}{Komponenten}
	
	Es sind insgesamt vier Hauptkomponenten zu erkennen:
	
	\begin{columns}
		\column{0.45\textwidth}
	\begin{itemize}
	\item das UserInterface
	\item der DataBaseManager
	\item die SearchEngine
	\item die Datenbank
	\end{itemize}

		\column{0.6\textwidth}
			\begin{figure}
				\includegraphics[width=\textwidth]{images/Architektur.png}
			\end{figure}
	\end{columns}

	\note{
		\begin{itemize}
			\item 
		\end{itemize}
	}

\end{frame}

\begin{frame} {Systemarchitektur}{Kommunikation der Komponenten}
	
	Die Kommunikation der Komponenten findet statt �ber
	
	\begin{itemize}
	\item HTTP: als Schnittstelle vom UserInterface zum dataBaseManager (REST-API)
	\item Unix-Sockets: zur Interprozesskommunikation zwischen dataBaseManager und searchEngine
	\item HTTP: als Schnittstelle vom dataBaseManager zur Datenbank
	\end{itemize}
	
	Die Schnittstellen sind klar definiert und die Kommunikation findet immer �ber den dataBaseManager statt.
	
	\skippingparagraph
	Das Datenformat der ausgetauschten Nachrichten ist JSON.
	
	

	\note{
		\begin{itemize}
			\item REST-API: Representational State Transfer: Die Bezeichnung ?Representational State Transfer? soll den �bergang vom aktuellen Zustand zum n�chsten Zustand (state) einer Applikation verbildlichen. Dieser Zustands�bergang erfolgt durch den Transfer der Daten, die den n�chsten Zustand repr�sentieren.
		\end{itemize}
	}

\end{frame}

\subsection{Programmiersysteme}

\begin{frame} {Systemarchitektur}{Auswahl der Programmiersysteme}
	
	F�r die Realisierung der einzelnen Komponenten wurden unterschiedliche Technologien verwendet:
	
	\begin{itemize}
	\item C/C++ f�r die searchEngine
		\begin{itemize}
		\item beste Performance f�r kritische Datenverarbeitung
		\item gute Parallelisierungsm�glichkeiten f�r Number Crunching
		\item gute Bibliotheksunterst�tzung in der Geo-Data Community
		\end{itemize}

	\item HTML/React f�r das userInterface
		\begin{itemize}
		\item Standard-Webtechnologien zur Realisierung des View
		\item Sehr gute, leicht zu bedienende Darstellungskomponenten
		\item Leichte Erweiterbarkeit und iterative Realisierungsm�glichkeit
		\item Vorwissen f�r GUI-Realisierung
		\end{itemize}

	\item Node.js/Javascript f�r den dataBaseManager
		\begin{itemize}
		\item Leichte Integration der Schnittstellen zu anderen Komponenten
		\item Native Unterst�tzung des JSON-Datenformats
		\item leichtgewichtiger Webserver
		\item sehr gute Unterst�tzung zum Interfacing der Datenbank
		\end{itemize}
	\end{itemize}
	\note{
		\begin{itemize}
			\item 
		\end{itemize}
	}

\end{frame}

\subsection{Erreichung der Architekturziele}

\begin{frame} {Systemarchitektur}{Architekturziele}
	
	Wurden die Architekturziele erreicht?
	\skippingparagraph
	
	\begin{tabular}{lc}
		\pause inh�rent parallel &  \textcolor{green}{\Checkmark}\\ 
		\pause lose gekoppelte Komponenten &  \textcolor{green}{\Checkmark}\\ 
		\pause einfache Einbindung externer Bibliotheken &  \textcolor{green}{\Checkmark}\\ 
		\pause einfache Erweiterbarkeit &  \textcolor{green}{\Checkmark}\\ 
		\pause Kommunikation �ber Standardprotokolle &  \textcolor{green}{\Checkmark}
	\end{tabular} 


	\note{
		\begin{itemize}
			\item 
		\end{itemize}
	}

\end{frame}


\section{Komponente "`searchEngine"'}

\subsection{Datenformate und Verarbeitung}
\subsubsection{GeoTIFF}
\subsubsection{Speicherformat: GeoJSON}
\subsection{GeoTIFFHandler}
\subsubsection{Datenextraktion}
\subsubsection{Koordinatensysteme}
\subsubsection{Das \texttt{GeoTIFFHandler}-Objekt}


\section{Komponente "'Durchmusterung"'}
\subsection{Durchmusterung der Geotiff Daten und Auffinden der Landebahnen}

\subsection{Interface der Library}

\subsection{Logischer Ablauf innerhalb der Landebahn-Erkennung}

\subsection{Initialisierung}

\subsection{Parallelverarbeitung mittels p\_threads zum Finden der Landebahnen}

\subsection{Detaillierte Beschreibung des Suchalgorithmus}

\subsection{Klassen und Objekte}

\subsection{Verwendete Datenstrukturen}

\subsubsection{Einfluss der Datenstrukturen auf die Performanz}

\subsection{Bestimmung des Speedups in Abh�ngigkeit des Parallelisierungsgrades}

\subsubsection{Erl�uterung des Messverfahren}

\subsubsection{Messergebnisse}

\subsubsection{Diskussion der Messergebnisse}

\subsection{Analyse mittels Profiler}

\subsection{Einfluss von Compiler Direktiven (speziell Optimizer)}

\section{Komponente "`databaseManager"'}

\subsection{Datenbankauswahl}

	\begin{frame} {Ein GeoJSON Objekt zur Ablage in die Datenbank}{}
			
		\smaller\lstinputlisting[style=JSStyle, linebackgroundcolor={%
			\btLstHL<2|handout:0>{18}%
			\btLstHL<3|handout:0>{19,20}% 
		}]{snippets/databaseObject.js}
		
	\end{frame}
	


	\subsection{Express-Server App}
	
	\subsection{Postprocessing - Merge}
	
	\section{Komponente "'landingClient"'}
	
	\begin{frame} {Benutzeroberfl�che}
		\begin{figure}[ht]
			\includegraphics[width=\textwidth]{../Bericht/drawings/LandingClient_Screen1.png}
			\caption{Beispiel Screenshot aus der Weboberfl�che zur Bedienung des Systems}
		\end{figure}
	\end{frame}
	
	\subsection{�berblick}
	
	\subsection{Features und Bedienung}
	
	\section{Ausblick und weitere Ideen}
	
	\subsection{Parallele Bearbeitung mehrerer Kacheln per MPI}
	
	\subsection{Richtungskorrektur des GeoTIFF}
	
	\subsection{Push f�r Webclient}
	
	\subsection{Merge als nachgeschalteter Prozess aus dem Speicher}
	
	\subsection{Kollisionsabfrage mit Objekten aus OSM}
	
\end{document}


