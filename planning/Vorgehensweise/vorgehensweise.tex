%----------------------------------------------------------------------------------------
%	PACKAGES AND OTHER DOCUMENT CONFIGURATIONS
%----------------------------------------------------------------------------------------

\documentclass[
11pt, % Main document font size
a4paper, % Paper type, use 'letterpaper' for US Letter paper
oneside, % One page layout (no page indentation)
%twoside, % Two page layout (page indentation for binding and different headers)
pdfspacing, % Makes use of pdftex’ letter spacing capabilities via the microtype package
headinclude,
%footinclude, % Extra spacing for the header and footer
BCOR5mm, % Binding correction
ngerman, %set document to German
bibtotocnumbered,
]{scrartcl}
%{scrartcl}

\input{structure.tex} % Include the structure.tex file which specified the document structure and layout
%----------------------------------------------------------------------------------------
%	TITLE AND AUTHOR(S)
%----------------------------------------------------------------------------------------
\selectlanguage{ngerman}

\subtitle{\normalfont{Fachpraktikum 1597 an der FernUni Hagen im SS 2017:\protect\\Parallele Programmierung }}

\title{\normalfont{Vorgehensweise zur Erkennung von möglichen Notlandefeldern aus Höhendaten}} % The article title
%\title{\normalfont\spacedallcaps{Asnychrone Programmierung: \protect\\ Moderne Methoden in ECMAScript~6}} % The article title



\author{Felix Eckstein*, Dr. Björn Wittlich**} % The article author(s) - author affiliations need to be specified in the AUTHOR AFFILIATIONS block

%----------------------------------------------------------------------------------------
%	AUTHOR AFFILIATIONS
%----------------------------------------------------------------------------------------

%\affil[*]{Student im Bachelor of Science Informatik, FernUniversität Hagen, Matr.-\#: 8161569, \href{mailto:felix.eckstein@gmx.de}{felix.eckstein@gmx.de}}


\date{Juni 2017} % An optional date to appear under the author(s)

%----------------------------------------------------------------------------------------

\begin{document}
	
	\maketitle % Print the title/author/date block

	
	{\let\thefootnote\relax\footnotetext{* \textit{Student im Bachelor of Science Informatik an der FernUniversität Hagen,\protect\\Matr.-\#: 8161569, \href{mailto:felix.eckstein@gmx.de}{felix.eckstein@gmx.de}}}}
	{\let\thefootnote\relax\footnotetext{** \textit{Student im xxx xxxxx Informatik an der FernUniversität Hagen,\protect\\Matr.-\#: xxx xxxx, \href{mailto:BjoernWittich@gmx.de}{BjoernWittich@gmx.de}}}}
	
	\selectlanguage{ngerman}

\section{Vorgehensweise zur Erkennung von möglichen Notlandefeldern aus Höhendaten}

	\subsection{Grundsätzliche Überlegungen}
	
	\subsection{Kriterien für geeignete Notlandeflächen}

\section{Datenbasis}

	\subsection{Auflösung}

\section{Durchmusterung}

\section{Datenhaltung gefundener Notlandeflächen}

\section{Benutzung und Visualisierung}

\section{Parallelisierung}

\end{document}