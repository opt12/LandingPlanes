%%%%%%%%%%%%%%%%%%%%%%%%%%%%%%%%%%%%%%%%%
% Arsclassica Article
% Structure Specification File
%
% This file has been downloaded from:
% http://www.LaTeXTemplates.com
%
% Original author:
% Lorenzo Pantieri (http://www.lorenzopantieri.net) with extensive modifications by:
% Vel (vel@latextemplates.com)
%
% License:
% CC BY-NC-SA 3.0 (http://creativecommons.org/licenses/by-nc-sa/3.0/)
%
%%%%%%%%%%%%%%%%%%%%%%%%%%%%%%%%%%%%%%%%%

%----------------------------------------------------------------------------------------
%	REQUIRED PACKAGES
%----------------------------------------------------------------------------------------

\usepackage[usenames,dvipsnames]{color} % Required for specifying custom colors and referring to colors by name

\usepackage[
nochapters, % Turn off chapters since this is an article        
beramono, % Use the Bera Mono font for monospaced text (\texttt)
%eulermath,% Use the Euler font for mathematics
eulerchapternumbers,
pdfspacing, % Makes use of pdftex’ letter spacing capabilities via the microtype package
dottedtoc % Dotted lines leading to the page numbers in the table of contents
]{classicthesis} % The layout is based on the Classic Thesis style

%\usepackage{arsclassica} % Modifies the Classic Thesis package

\usepackage[T1]{fontenc} % Use 8-bit encoding that has 256 glyphs

\usepackage[utf8]{inputenc} % Required for including letters with accents

\usepackage{graphicx} % Required for including images
\graphicspath{{Figures/}} % Set the default folder for images

\usepackage{enumitem} % Required for manipulating the whitespace between and within lists

\usepackage{lipsum} % Used for inserting dummy 'Lorem ipsum' text into the template

\usepackage{amsmath,amssymb,amsthm} % For including math equations, theorems, symbols, etc

\usepackage{varioref} % More descriptive referencing

\usepackage{url}

\usepackage{hyperref}

\usepackage{gensymb}	%for the \degree command

\usepackage[ngerman, english]{babel}

\usepackage[square,
authoryear,
%numbers,
longnamesfirst
]{natbib}

\usepackage{enumitem}

\usepackage{authblk}

\usepackage{relsize, etoolbox, lmodern}% http://ctan.org/pkg/{relsize,etoolbox}

\AtBeginEnvironment{quote}{\smaller\fontfamily{lmss}\selectfont}% Step font down one size relative to current font.
% see http://tex.stackexchange.com/questions/25249/how-do-i-use-a-particular-font-for-a-small-section-of-text-in-my-document for fontfamilies

\usepackage{chngcntr}

% to have a separation in the text with extra space between paragraphs and no indendation following
\newcommand*{\skippingparagraph}{\par\vspace{1.0\baselineskip}\noindent}

%----------------------------------------------------------------------------------------
%	HYPERLINKS
%---------------------------------------------------------------------------------------
\definecolor{mediumviolet-red}{rgb}{0.78, 0.08, 0.52}
\hypersetup{
%	draft, % Uncomment to remove all links (useful for printing in black and white)
	colorlinks=true, breaklinks=true, bookmarks=true,bookmarksnumbered,
	urlcolor=webbrown, linkcolor=mediumviolet-red, citecolor=webgreen, % Link colors
	pdftitle={}, % PDF title
	pdfauthor={\textcopyright}, % PDF Author
	pdfsubject={}, % PDF Subject
	pdfkeywords={}, % PDF Keywords
	pdfcreator={pdfLaTeX}, % PDF Creator
	pdfproducer={LaTeX with hyperref and ClassicThesis} % PDF producer
}

%----------------------------------------------------------------------------------------
%	TODONOTES
%---------------------------------------------------------------------------------------

\usepackage[colorinlistoftodos]{todonotes}

\newcommand{\todoInfo}[1]{\todo[color=blue!25]{INFO: #1}}
\newcommand{\todoCite}[1]{\todo[color=green!40]{INFO: #1}}


%----------------------------------------------------------------------------------------
%	CODESNIPPETS
%---------------------------------------------------------------------------------------

%%%%%%%%%%%%%%%%%%%%%%%%%%%%%%%%%%%%%%%%%
% Code Snippet
% LaTeX Template
% Version 1.0 (14/2/13)
%
% This template has been downloaded from:
% http://www.LaTeXTemplates.com
%
% Original author:
% Velimir Gayevskiy (vel@latextemplates.com)
%
% License:
% CC BY-NC-SA 3.0 (http://creativecommons.org/licenses/by-nc-sa/3.0/)
%
%%%%%%%%%%%%%%%%%%%%%%%%%%%%%%%%%%%%%%%%%
\usepackage{listings} % Required for inserting code snippets

\definecolor{DarkGreen}{rgb}{0.0,0.4,0.0} % Comment color
\definecolor{highlight}{RGB}{255,251,204} % Code highlight color
\definecolor{DogwoodRose}{rgb}{0.84, 0.09, 0.41}
\definecolor{lemonchiffon}{rgb}{1.0, 0.98, 0.8}
\definecolor{lightgoldenrodyellow}{rgb}{0.98, 0.98, 0.82}
\definecolor{oldlace}{rgb}{0.99, 0.96, 0.9}
\definecolor{oldlavender}{rgb}{0.47, 0.41, 0.47}
\definecolor{pakistangreen}{rgb}{0.0, 0.4, 0.0}
\definecolor{ao}{rgb}{0.0, 0.0, 1.0}
\definecolor{britishracinggreen}{rgb}{0.0, 0.26, 0.15}
\definecolor{coquelicot}{rgb}{1.0, 0.22, 0.0}
\definecolor{cordovan}{rgb}{0.54, 0.25, 0.27}
\definecolor{darkcandyapplered}{rgb}{0.64, 0.0, 0.0}
\definecolor{mediumchampagne}{rgb}{0.95, 0.9, 0.67}

\lstdefinestyle{Style1}{ % Define a style for your code snippet, multiple definitions can be made if, for example, you wish to insert multiple code snippets using different programming languages into one document
	language=JavaScript, % Detects keywords, comments, strings, functions, etc for the language specified
	backgroundcolor=\color{oldlace}, % Set the background color for the snippet - useful for highlighting
	basicstyle=\smaller\smaller\ttfamily, % The default font size and style of the code
	breakatwhitespace=true, % If true, only allows line breaks at white space
	breaklines=true, % Automatic line breaking (prevents code from protruding outside the box)
	captionpos=b, % Sets the caption position: b for bottom; t for top
	commentstyle=\usefont{T1}{pcr}{m}{sl}\color{ao}, % {m}DarkGreen Style of comments within the code - dark green courier font
	deletekeywords={}, % If you want to delete any keywords from the current language separate them by commas
	%escapeinside={\%}, % This allows you to escape to LaTeX using the character in the bracket
	firstnumber=1, % Line numbers begin at line 1
	frame=single, % Frame around the code box, value can be: none, leftline, topline, bottomline, lines, single, shadowbox
	frameround=tttt, % Rounds the corners of the frame for the top left, top right, bottom left and bottom right positions
	keywordstyle=\color{darkcandyapplered}\textbf, % Functions are bold and blue
	morekeywords={}, % Add any functions no included by default here separated by commas
	numbers=left, % Location of line numbers, can take the values of: none, left, right
	numbersep=10pt, % Distance of line numbers from the code box
	numberstyle=\tiny\color{Gray}, % Style used for line numbers
	rulecolor=\color{black}, % Frame border color
	showstringspaces=false, % Don't put marks in string spaces
	showtabs=false, % Display tabs in the code as lines
	stepnumber=5, % The step distance between line numbers, i.e. how often will lines be numbered
	stringstyle=\color{Purple}, % Strings are purple
	tabsize=2, % Number of spaces per tab in the code
}

\definecolor{darkgray}{rgb}{.4,.4,.4}
\definecolor{purple}{rgb}{0.65, 0.12, 0.82}


%define Javascript language
\lstdefinelanguage{JavaScript}{
	keywords={typeof, new, true, false, catch, try, finally, function, return, null, then, catch, switch, var, if, in, while, do, else, case, break, yield, async, await},
	keywordstyle=\color{blue}\bfseries,
	ndkeywords={class, export, boolean, throw, implements, import, this},
	ndkeywordstyle=\color{darkgray}\bfseries,
	identifierstyle=\color{black},
	sensitive=false,
	comment=[l]{//},
	morecomment=[s]{/*}{*/},
	commentstyle=\color{purple}\ttfamily,
	stringstyle=\color{red}\ttfamily,
	morestring=[b]',
	morestring=[b]"
}

\lstset{
	language=JavaScript,
	extendedchars=true,
	basicstyle=\footnotesize\ttfamily,
	showstringspaces=false,
	showspaces=false,
	numbers=left,
	numberstyle=\footnotesize,
	numbersep=9pt,
	tabsize=2,
	breaklines=true,
	showtabs=false,
	captionpos=b
}

% Create a command to cleanly insert a snippet with the style above anywhere in the document
\newcommand{\insertcode}[2]{\begin{itemize}\item[]\lstinputlisting[{caption={#2}},label=#1,style=Style1]{#1}\end{itemize}} % The first argument is the script location/filename and the second is a caption for the listing
